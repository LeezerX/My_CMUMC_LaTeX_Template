\documentclass{cumcmthesis}
    %指定使用该模板, 需将cumcmthesis.cls文件于tex文件置于同一目录下
    %  可选参数:
    %       - withoutpreface:提交电子版时,无需承诺书和封面
    %       - bwprint:黑白打印,默认为colorprint

\title{论文题目}
\tihao{A}            % 题号
\baominghao{4321}    % 报名号
\schoolname{你的大学}
\membera{成员A}
\memberb{成员B}
\memberc{成员C}
\supervisor{指导老师}
\yearinput{2017}     % 年
\monthinput{08}      % 月
\dayinput{22}        % 日

\begin{document}

\maketitle % 承诺书与封面页

% 可根据个人需求,修改文章主体    

% 摘要
\begin{abstract}

    摘要的具体内容。
    \keywords{关键词1\quad  关键词2\quad   关键词3}

\end{abstract}

\tableofcontents %目录, 不需要目录时, 注释掉即可

\section{问题重述}

    \subsection{问题背景}

    \subsection{目标任务}

\section{问题分析}

\section{模型假设}

\section{符号说明}
    \begin{table}[!h]
        \centering
        \caption{符号说明}
        \setlength\tabcolsep{60pt} %修改最左边一列到边栏的间距
        \begin{tabularx}{\textwidth}{XX}
            \toprule[1.5pt]
            符号 & 意义\\
            \midrule[1pt]
             & \\
            \bottomrule[1.5pt]
        \end{tabularx}
    \end{table}

\section{模型的建立与求解}

\section{灵敏度分析}

\section{模型的评价与推广}

% 参考文献
\begin{thebibliography}{99}%宽度99

    \bibitem[1]{1}

\end{thebibliography}

% 附录
\begin{appendices}

\section{代码}
%\lstinputlisting[language=, firstline=, lastline=,caption=]{source}

\end{appendices}

\end{document}